% !TeX program = pdfLaTeX
\documentclass[smallextended]{svjour3}       % onecolumn (second format)
%\documentclass[twocolumn]{svjour3}          % twocolumn
%
\smartqed  % flush right qed marks, e.g. at end of proof
%
\usepackage{amsmath,amssymb}
\usepackage{graphicx}
\usepackage[utf8]{inputenc}

\usepackage[hyphens]{url} % not crucial - just used below for the URL
\usepackage{hyperref}

% common operators
\newcommand*{\der}[2]{\frac{\mathrm{d}#1}{\mathrm{d}#2}}

% structural markers
\def\div{\;|\;}
\def\bigdiv{\;\big|\;}
\def\Bigdiv{\;\Big|\;}
\def\Biggdiv{\;\Bigg|\;}

%
% \usepackage{mathptmx}      % use Times fonts if available on your TeX system
%
% insert here the call for the packages your document requires
%\usepackage{latexsym}
% etc.
%
% please place your own definitions here and don't use \def but
% \newcommand{}{}
%
% Insert the name of "your journal" with
% \journalname{myjournal}
%

%% load any required packages here



% tightlist command for lists without linebreak
\providecommand{\tightlist}{%
  \setlength{\itemsep}{0pt}\setlength{\parskip}{0pt}}



\begin{document}


\title{Title here \thanks{Grants or other notes about the article that
should go on the front page should be placed here. General
acknowledgments should be placed at the end of the article.} }
 \subtitle{Do you have a subtitle? If so, write it here} 

    \titlerunning{Short form of title (if too long for head)}

\author{  Äüthör 1 \and  Âuthóř 2 \and  }

    \authorrunning{ Short form of author list if too long for running
head }

\institute{
        Äüthör 1 \at
     Department of YYY, University of XXX \\
     \email{\href{mailto:abc@def}{\nolinkurl{abc@def}}}  %  \\
%             \emph{Present address:} of F. Author  %  if needed
    \and
        Âuthóř 2 \at
     Department of ZZZ, University of WWW \\
     \email{\href{mailto:djf@wef}{\nolinkurl{djf@wef}}}  %  \\
%             \emph{Present address:} of F. Author  %  if needed
    \and
    }

\date{Received: date / Accepted: date}
% The correct dates will be entered by the editor


\maketitle

\begin{abstract}
The text of your abstract. 150 -- 250 words.
\\
\keywords{
        key \and
        dictionary \and
        word \and
    }

    \subclass{
                    MSC code 1 \and
                    MSC code 2 \and
            }

\end{abstract}


\def\spacingset#1{\renewcommand{\baselinestretch}%
{#1}\small\normalsize} \spacingset{1}


\hypertarget{intro}{%
\section{Introduction}\label{intro}}

Your text comes here. Separate text sections with \cite{Mislevy06Cog}.

\hypertarget{sec:1}{%
\section{Background}\label{sec:background}}

% Text with citations by \cite{Galyardt14mmm}.
In this section we first review two well-established techniques commonly used in sustainable fishery management. 
These are the maximum sustainable yield (MSY) and the constant escapement (CE) approaches. 
After this, deep reinforcement learning is briefly reviewed

\hypertarget{sec:2}{%
\subsection{Fishery management}\label{sec:fishery}}

\hypertarget{sec:3}{%
\subsection{Deep reinforcement learning}\label{sec:rl}}

\hypertarget{sec:4}{%
\section{Dynamical models used}\label{sec:model}}

In this section we present three models of increasing complexity which plausibly describe the population dynamics of a marine ecosystem.
These models will form the test beds for the comparison between classical fishery management strategies and DRL.

\hypertarget{sec:4.1}{%
\subsection{A one-dimensional tipping point model.}\label{p:may}}

Consider a population $V$ whose dynamics is given by
\begin{align}
  \label{eq:may77}
  \der{V}{t} = rV\left(
    1-V/K
    \right) 
    -
    \frac{\beta H V^2}{V_0^2 + V^2}. 
\end{align}
This model has been used in~\cite{may77} to describe a grazing ecosystem, where a species $V$ of vegetation is harvested by a constant herbivore population $H$. 
% Here the only dynamic degree of freedom is $V$, while $r,\ K,\ \beta,\ V_0,$ and $H > 0$ are fixed positive constants.

In~\eqref{eq:may77}, a population $V$ grows logistically with rate $r$ up to carrying capacity $K$.
This is expressed by the first term in the equation,
\begin{align*}
  L(V \div r,\ K) := rV\left(1 - V / K\right).
\end{align*}
Moreover, $V$ is predated on by a (constant) population $H$, as can be seen from the negative term
\begin{align*}
  F(V,\ H \div \beta,\ V_0) := \frac{\beta H V^2}{V_0^2 + V^2}
\end{align*}
which saturates to $\beta H$ as $V\to\infty$, and whose half maximum is $V_0$, i.e.\ $F(V=V_0,\ H; \beta,\ V_0)=\beta H/2$.

Ref.~\cite{may77} studies the fixed points of~\eqref{eq:may77} in order to show that in certain parameter regimes, its dynamics can undergo a \emph{catastrophe}.
A catastrophe is a sudden change in the state of the system from one stable state to another---often, the final state is ecologically detrimental, possibly associated with extinction or near-extinction events.

Fig.~\ref{fig:tbd} shows the stable $V$ populations for differing values of $H$. 
Here one sees that as $H\to T_2$, the top stable state is annihilated with the unstable fixed point.
This way, if, e.g.\ $H$ were to slowly drift until $H=T_2$, the system would collapse to the low stable state, leading to a near extinction of $V$.


% For fixed values of $r,\ K,\ V_0$ and $\beta$, varying $H$ may produce a \emph{catastrophe} (see Fig.~\ref{fig:tbd})---a sudden disappearance of a stable fixed point of the system which . 

\hypertarget{sec:5}{%
\section{Results}\label{sec:results}}

\hypertarget{sec:6}{%
\section{Discussion}\label{sec:discussion}}




\hypertarget{paragraph-headings}{%
\paragraph{Paragraph headings}\label{paragraph-headings}}

Use paragraph headings as needed. 
Use paragraph headings as needed. 
Use paragraph headings as needed. 
Use paragraph headings as needed. 
Use paragraph headings as needed. 
Use paragraph headings as needed. 
Use paragraph headings as needed. 

\begin{align}
a^2+b^2=c^2
\end{align}


\bibliographystyle{spphys}
\bibliography{bibliography.bib}


\end{document}
